\documentclass[12pt]{article}

% Специальный шрифт, чтобы лучше читалось
\usepackage{libertine}

\usepackage{fullpage}

\usepackage{multicol,multirow,csquotes}
\usepackage{tabularx}
\usepackage{ulem}
\usepackage[utf8]{inputenc}
\usepackage[russian]{babel}
\usepackage{amsmath}
\usepackage{amssymb}

\usepackage{titlesec}

\titleformat{\section}
  {\normalfont\Large\bfseries}{\thesection.}{0.3em}{}

\titleformat{\subsection}
  {\normalfont\large\bfseries}{\thesubsection.}{0.3em}{}

\titlespacing{\section}{0pt}{*2}{*2}
\titlespacing{\subsection}{0pt}{*1}{*1}
\titlespacing{\subsubsection}{0pt}{*0}{*0}
\usepackage{listings}
\lstloadlanguages{Lisp}
\lstset{extendedchars=false,
	breaklines=true,
	breakatwhitespace=true,
	keepspaces = true,
	tabsize=8
}
\begin{document}

\section*{Отчет по лабораторной работе № 4 \\
по курсу \guillemotleft Функциональное программирование\guillemotright}
\begin{flushright}
Студент группы М8О-307-19 МАИ \textit{Ежов Никита Павлович}, \textnumero 9 по списку \\
\makebox[7cm]{Контакты: {\tt nikita.ejov2012@yandex.ru} \hfill} \\
\makebox[7cm]{Работа выполнена: 25.05.2022 \hfill} \\
\ \\
Преподаватель: Иванов Дмитрий Анатольевич, доц. каф. 806 \\
\makebox[7cm]{Отчет сдан: \hfill} \\
\makebox[7cm]{Итоговая оценка: \hfill} \\
\makebox[7cm]{Подпись преподавателя: \hfill} \\

\end{flushright}

\section{Тема работы}
Последовательности, массивы и управляющие конструкции Common Lisp.

\section{Цель работы}
Научиться работать с литерами (знаками) и строками при помощи функций обработки строк и общих функций работы с последовательностями.

\section{Задание (вариант № 4.13)}
Запрограммировать на языке Коммон Лисп функцию, принимающую один аргумент - предложение.

Функция должна возвращать копию исходного предложения, в которой из каждого слова удалена первая литера. Полученные пустые слова отбрасываются и оставшиеся непустые слова должны разделяться лишь одним пробелом.

(trim-first-char "это  ушла срезка  с плугом") => "то шла резка лугом"

\section{Оборудование студента}
Процессор Intel i7-4770 (8) @ 3.9GHz, память: 16 Gb, разрядность системы: 64.

\section{Программное обеспечение}
ОС Kubuntu 20.04.4 LTS, комилятор GNU CLISP 2.49.92, текстовый редактор Visual Studio Code 1.67.1

\pagebreak
\section{Идея, метод, алгоритм}
Мы разбиваем строку на слова, при этом удаляя из этих слов первый символ. Потом обратно склеливаем слова в конечную строку и
возвращаем её.

\section{Сценарий выполнения работы}

\section{Распечатка программы и её результаты}

\subsection{Исходный код}
\lstinputlisting{./Ezhov_307_Lab4.lisp}

\pagebreak
\subsection{Результаты работы}
[1]> (trim-first-char \enquote{это  ушла срезка  с плугом})


\enquote{то шла резка лугом}

\pagebreak
\section{Дневник отладки}
\begin{tabular}{|p{50pt}|p{140pt}|p{140pt}|p{80pt}|}
\hline
Дата & Событие & Действие по исправлению & Примечание \\
\hline
\end{tabular}

\section{Замечания автора по существу работы}

Сложность полученного решения $O(n)$. Пространственная сложность тоже $O(n)$, поэтому добиться лучшей асимптотики нельзя.

\section{Выводы}
Я познакомился со строками и литерами в Common Lisp. Очень приятно использовать высокоуровневые функции, например subseq,
встроенные в язык. Чем-то напоминает Python, в котором так же легко работать со строками

\end{document}
