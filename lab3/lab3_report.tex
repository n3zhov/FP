\documentclass[12pt]{article}

% Специальный шрифт, чтобы лучше читалось
\usepackage{libertine}

\usepackage{fullpage}
\usepackage{multicol,multirow}
\usepackage{tabularx}
\usepackage{ulem}
\usepackage[utf8]{inputenc}
\usepackage[russian]{babel}
\usepackage{amsmath}
\usepackage{amssymb}

\usepackage{titlesec}

\titleformat{\section}
  {\normalfont\Large\bfseries}{\thesection.}{0.3em}{}

\titleformat{\subsection}
  {\normalfont\large\bfseries}{\thesubsection.}{0.3em}{}

\titlespacing{\section}{0pt}{*2}{*2}
\titlespacing{\subsection}{0pt}{*1}{*1}
\titlespacing{\subsubsection}{0pt}{*0}{*0}
\usepackage{listings}
\lstloadlanguages{Lisp}
\lstset{extendedchars=false,
	breaklines=true,
	breakatwhitespace=true,
	keepspaces = true,
	tabsize=8
}
\begin{document}

\section*{Отчет по лабораторной работе № 3 \\
по курсу \guillemotleft Функциональное программирование\guillemotright}
\begin{flushright}
Студент группы М8О-307-19 МАИ \textit{Ежов Никита Павлович}, \textnumero 9 по списку \\
\makebox[7cm]{Контакты: {\tt nikita.ejov2012@yandex.ru} \hfill} \\
\makebox[7cm]{Работа выполнена: 08.05.2022 \hfill} \\
\ \\
Преподаватель: Иванов Дмитрий Анатольевич, доц. каф. 806 \\
\makebox[7cm]{Отчет сдан: \hfill} \\
\makebox[7cm]{Итоговая оценка: \hfill} \\
\makebox[7cm]{Подпись преподавателя: \hfill} \\

\end{flushright}

\section{Тема работы}
Последовательности, массивы и управляющие конструкции Common Lisp.

\section{Цель работы}
Научиться создавать векторы и массивы для представления матриц, освоить общие функции работы с последовательностями, инструкции цикла и нелокального выхода.

\section{Задание (вариант № 3.20)}
Запрограммировать на языке Коммон Лисп функцию, принимающую в качестве аргумента двумерный массив, представляющий действительную матрицу произвольного размера.

Функция должна возвращать новую матрицу того же размера, получающуюся из данной перестановкой строк - первой с последней, второй с предпоследней и т.д.

Исходный массив должен оставаться неизменным.

\section{Оборудование студента}
Процессор Intel i7-4770 (8) @ 3.9GHz, память: 16 Gb, разрядность системы: 64.

\section{Программное обеспечение}
ОС Kubuntu 20.04.4 LTS, комилятор GNU CLISP 2.49.92, текстовый редактор Visual Studio Code 1.67.1

\pagebreak
\section{Идея, метод, алгоритм}
Скопирую матрицу, чтобы не менять исходную. Затем сохраню в отдельные переменные кол-во столбцов и строки скопированной матрицы.

Обход выполняется следующим образом: мы проходим по всем строчкам до $n1 // 2$ (целочисленное деление) и меняем элементы $i$-й строки с элементами
$m1 - i$ строки. В программе обход реализован следующим образом: если $mod(m1, 2) == 1$, то мы отнимаем от $m1$ единицу, чтобы оно стало чётным числом.
Потом запескаем цикл по строкам от нулевой и до $(m1 / 2) - 1$, меняя строки местами поэлементно, как это описано выше. Таким образом и реализована 
перестановка первой строки с последней, второй с предпоследней и т.д.

\section{Сценарий выполнения работы}

\section{Распечатка программы и её результаты}

\subsection{Исходный код}
\lstinputlisting{./lab3.lisp}

\pagebreak
\subsection{Результаты работы}
\lstinputlisting{./log3.txt}

\pagebreak
\section{Дневник отладки}
\begin{tabular}{|p{50pt}|p{140pt}|p{140pt}|p{80pt}|}
\hline
Дата & Событие & Действие по исправлению & Примечание \\
\hline
\end{tabular}

\section{Замечания автора по существу работы}

Сложность полученного решения $O(n \cdot m)$. Пространственная сложность тоже $O(n \cdot m)$, поэтому добиться лучшей асимптотики нельзя.

\section{Выводы}
Я познакомился с векторами и массивами в Common Lisp. Язык поддерживает императивную парадигму в отличии от, например, Prolog, на котором работа с матрицами очень неприятная и сложная.

\end{document}
