\documentclass[12pt]{article}

% Специальный шрифт, чтобы лучше читалось
\usepackage{libertine}

\usepackage{fullpage}
\usepackage{multicol,multirow}
\usepackage{tabularx}
\usepackage{ulem}
\usepackage[utf8]{inputenc}
\usepackage[russian]{babel}
\usepackage{amsmath}
\usepackage{amssymb}

\usepackage{titlesec}

\titleformat{\section}
  {\normalfont\Large\bfseries}{\thesection.}{0.3em}{}

\titleformat{\subsection}
  {\normalfont\large\bfseries}{\thesubsection.}{0.3em}{}

\titlespacing{\section}{0pt}{*2}{*2}
\titlespacing{\subsection}{0pt}{*1}{*1}
\titlespacing{\subsubsection}{0pt}{*0}{*0}
\usepackage{listings}
\lstloadlanguages{Lisp}
\lstset{extendedchars=false,
	breaklines=true,
	breakatwhitespace=true,
	keepspaces = true,
	tabsize=2
}
\begin{document}

\section*{Отчет по лабораторной работе № 2 \\
по курсу \guillemotleft Функциональное программирование\guillemotright}
\begin{flushright}
Студент группы М8О-307Б-19 МАИ \textit{Ежов Никита Павлович}, \textnumero 9 по списку \\
\makebox[7cm]{Контакты: {\tt nikita.ejov2012@yandex.ru} \hfill} \\
\makebox[7cm]{Работа выполнена: 12.04.2022 \hfill} \\
\ \\
Преподаватель: Иванов Дмитрий Анатольевич, доц. каф. 806 \\
\makebox[7cm]{Отчет сдан: \hfill} \\
\makebox[7cm]{Итоговая оценка: \hfill} \\
\makebox[7cm]{Подпись преподавателя: \hfill} \\

\end{flushright}

\section{Тема работы}
Простейшии функции работы со списками Common Lisp.

\section{Цель работы}
Научиться конструировать списки, находить элемент в списке, использовать схему линейной и древовидной рекурсии для обхода и реконструкции
плоских списков и деревьев.

\section{Задание (вариант № 2.15)}
Запрограммируйте рекурсивно на языке Коммон Лисп функцию, подсчитывающую число вхождений заданного действительного числа в дерево. 
"Действительное" означает, что типы чисел могут смешиваться: целые, рациональные и с плавающей точкой.

\section{Оборудование студента}
Процессор Intel i7-4770 (8) @ 3.9GHz, память: 16 Gb, разрядность системы: 64.

\section{Программное обеспечение}
ОС Kubuntu 20.04.4 LTS, комилятор GNU CLISP 2.49.92, текстовый редактор Atom 1.58.0

\pagebreak
\section{Идея, метод, алгоритм}
Реализую алгоритм обхода дерева при помощи древовидной рекурсии, и каждый лист буду сравнивать с заданным в аргументе функции числом.
При обходе вглубь буду также передавать число, с которым нужно сравнить лист. Если описывать написанную программу, то у нас
существует функция, которая возвращает 0 в случае если данный узел не является листом или если значения листа не равно заданному аргументу
и 1 в случае, если значение листа равно заданному аргументу. В качестве ответа получается сумма всех резултатов вызова функции.

\section{Сценарий выполнения работы}

\section{Распечатка программы и её результаты}

\subsection{Исходный код}
\lstinputlisting{./lab2.lisp}

\pagebreak
\subsection{Результаты работы}
\lstinputlisting{./log2.txt}

\pagebreak
\section{Дневник отладки}
\begin{tabular}{|p{50pt}|p{80pt}|p{140pt}|p{140pt}|}
\hline
Дата & Событие & Действие по исправлению & Примечание \\
\hline
18.04.2022 & Написанная программа работает некорректно, если ей передавать в качестве аргумента любое число кроме 3 &
Был обнаружен код, использовавшийся для отладки на конкретном примере и замёнен на такой, который работает с переданным аргументом,
а не с числом 3 & Ошибка была обнаружена только после замечания преподавателя \\
\hline
\end{tabular}

\section{Замечания автора по существу работы}
Из-за древовидной рекурсии возможно быстрое увелечение потребляемой памяти при сильной "ветвистости" анализируемого дерева.

\section{Выводы}
Я познакомился со списками и деревьями в Common Lisp. Работа с ними очень похожа на работу с аналогичными структурами в языке Prolog,
поэтому работу было выполнить несложно.

\end{document}